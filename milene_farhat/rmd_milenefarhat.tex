\PassOptionsToPackage{unicode=true}{hyperref} % options for packages loaded elsewhere
\PassOptionsToPackage{hyphens}{url}
%
\documentclass[]{article}
\usepackage{lmodern}
\usepackage{amssymb,amsmath}
\usepackage{ifxetex,ifluatex}
\usepackage{fixltx2e} % provides \textsubscript
\ifnum 0\ifxetex 1\fi\ifluatex 1\fi=0 % if pdftex
  \usepackage[T1]{fontenc}
  \usepackage[utf8]{inputenc}
  \usepackage{textcomp} % provides euro and other symbols
\else % if luatex or xelatex
  \usepackage{unicode-math}
  \defaultfontfeatures{Ligatures=TeX,Scale=MatchLowercase}
\fi
% use upquote if available, for straight quotes in verbatim environments
\IfFileExists{upquote.sty}{\usepackage{upquote}}{}
% use microtype if available
\IfFileExists{microtype.sty}{%
\usepackage[]{microtype}
\UseMicrotypeSet[protrusion]{basicmath} % disable protrusion for tt fonts
}{}
\IfFileExists{parskip.sty}{%
\usepackage{parskip}
}{% else
\setlength{\parindent}{0pt}
\setlength{\parskip}{6pt plus 2pt minus 1pt}
}
\usepackage{hyperref}
\hypersetup{
            pdftitle={Atividade - Inteligência de Dados},
            pdfauthor={Milene V. Farhat},
            pdfborder={0 0 0},
            breaklinks=true}
\urlstyle{same}  % don't use monospace font for urls
\usepackage[margin=1in]{geometry}
\usepackage{longtable,booktabs}
% Fix footnotes in tables (requires footnote package)
\IfFileExists{footnote.sty}{\usepackage{footnote}\makesavenoteenv{longtable}}{}
\usepackage{graphicx,grffile}
\makeatletter
\def\maxwidth{\ifdim\Gin@nat@width>\linewidth\linewidth\else\Gin@nat@width\fi}
\def\maxheight{\ifdim\Gin@nat@height>\textheight\textheight\else\Gin@nat@height\fi}
\makeatother
% Scale images if necessary, so that they will not overflow the page
% margins by default, and it is still possible to overwrite the defaults
% using explicit options in \includegraphics[width, height, ...]{}
\setkeys{Gin}{width=\maxwidth,height=\maxheight,keepaspectratio}
\setlength{\emergencystretch}{3em}  % prevent overfull lines
\providecommand{\tightlist}{%
  \setlength{\itemsep}{0pt}\setlength{\parskip}{0pt}}
\setcounter{secnumdepth}{0}
% Redefines (sub)paragraphs to behave more like sections
\ifx\paragraph\undefined\else
\let\oldparagraph\paragraph
\renewcommand{\paragraph}[1]{\oldparagraph{#1}\mbox{}}
\fi
\ifx\subparagraph\undefined\else
\let\oldsubparagraph\subparagraph
\renewcommand{\subparagraph}[1]{\oldsubparagraph{#1}\mbox{}}
\fi

% set default figure placement to htbp
\makeatletter
\def\fps@figure{htbp}
\makeatother


\title{Atividade - Inteligência de Dados}
\author{Milene V. Farhat}
\date{22/08/2021}

\begin{document}
\maketitle

\hypertarget{exercuxedcio-1}{%
\subsubsection{Exercício 1}\label{exercuxedcio-1}}

\textbf{1.1} Acesse o endpoint \texttt{deputados} e: \emph{Faça uma
requisição para obter a lista de deputados na atual legislatura em ordem
alfabética. }Transforme o resultado dessa requisição em um \emph{tibble}
de 9 colunas e 512 linhas com as informações básicas sobre os deputados.
\emph{Salve o dataset resultante em um CSV com o nome
}resposta-exercicio-1.csv*.

\begin{verbatim}
## # A tibble: 6 x 9
##       id uri   nome  siglaPartido uriPartido siglaUf idLegislatura urlFoto email
##    <int> <chr> <chr> <chr>        <chr>      <chr>           <int> <chr>   <chr>
## 1 204554 http~ Abíl~ PL           https://d~ BA                 56 https:~ dep.~
## 2 204521 http~ Abou~ PSL          https://d~ SP                 56 https:~ dep.~
## 3 204379 http~ Acác~ PROS         https://d~ AP                 56 https:~ dep.~
## 4 204560 http~ Adol~ PSDB         https://d~ BA                 56 https:~ dep.~
## 5 204528 http~ Adri~ NOVO         https://d~ SP                 56 https:~ dep.~
## 6 121948 http~ Adri~ PP           https://d~ GO                 56 https:~ dep.~
\end{verbatim}

\textbf{1.2} A partir do dataset recém-criado, conte o número de
deputados por partido e o número de deputados por estado.

\hypertarget{nuxfamero-de-deputados-por-estado}{%
\section{``Número de deputados por
estado''}\label{nuxfamero-de-deputados-por-estado}}

\begin{verbatim}
## # A tibble: 27 x 2
##    estado nr_deputados
##    <chr>         <int>
##  1 AC                8
##  2 AL                9
##  3 AM                8
##  4 AP                8
##  5 BA               39
##  6 CE               22
##  7 DF                8
##  8 ES               10
##  9 GO               17
## 10 MA               18
## # ... with 17 more rows
\end{verbatim}

\begin{longtable}[]{@{}lr@{}}
\caption{Número de deputados por partido}\tabularnewline
\toprule
partido & nr\_deputados\tabularnewline
\midrule
\endfirsthead
\toprule
partido & nr\_deputados\tabularnewline
\midrule
\endhead
AVANTE & 8\tabularnewline
CIDADANIA & 7\tabularnewline
DEM & 27\tabularnewline
MDB & 34\tabularnewline
NOVO & 8\tabularnewline
PATRIOTA & 6\tabularnewline
PCdoB & 8\tabularnewline
PDT & 25\tabularnewline
PL & 41\tabularnewline
PODE & 10\tabularnewline
PP & 41\tabularnewline
PROS & 11\tabularnewline
PSB & 31\tabularnewline
PSC & 11\tabularnewline
PSD & 34\tabularnewline
PSDB & 33\tabularnewline
PSL & 53\tabularnewline
PSOL & 9\tabularnewline
PT & 53\tabularnewline
PTB & 10\tabularnewline
PV & 4\tabularnewline
REDE & 1\tabularnewline
REPUBLICANOS & 32\tabularnewline
S.PART. & 1\tabularnewline
SOLIDARIEDADE & 14\tabularnewline
\bottomrule
\end{longtable}

\hypertarget{exercuxedcio-2}{%
\subsubsection{Exercício 2}\label{exercuxedcio-2}}

\textbf{2.1} Transforme esse conjunto de dados de forma que exista
apenas uma coluna que concentre as palavras-chave (chamada
\texttt{keyword}) e apenas uma coluna que concentre o gasto em dólares
(chamada \texttt{spend\_usd}). Salve o dataset resultante em um CSV com
o nome \emph{resposta-exercicio-2.csv}.

\begin{verbatim}
##    election_cycle report_date region  elections          keyword spend_usd
## 1 US-Federal-2018  2019-12-02     US US-Federal elizabeth warren   1491100
## 2 US-Federal-2018  2019-11-25     US US-Federal elizabeth warren   1479900
## 3 US-Federal-2018  2019-11-20     US US-Federal elizabeth warren   1469600
## 4 US-Federal-2018  2019-11-18     US US-Federal elizabeth warren   1466600
## 5 US-Federal-2018  2019-11-11     US US-Federal elizabeth warren   1452300
## 6 US-Federal-2018  2019-11-04     US US-Federal elizabeth warren   1430200
\end{verbatim}

\textbf{2.2} A partir do dataset recém-criado, calcule o valor médio por
palavra-chave em todo o período e descreva qual as 3 palavras-chave com
maior valor gasto.

As 3 palavras-chave com maior valor gasto são: \textbf{donald trump},
\textbf{tulsi gabbard}, \textbf{kamala harris}, com valor médio gasto de
\textbf{US\$ 914.755,00}, \textbf{US\$ 816.430,56} e \textbf{US\$
712.754,49}, respectivamente.

\begin{longtable}[]{@{}ll@{}}
\toprule
keyword & avg\_spend\_usd\tabularnewline
\midrule
\endhead
donald trump & US\$ 914.755,00\tabularnewline
tulsi gabbard & US\$ 816.430,56\tabularnewline
kamala harris & US\$ 712.754,49\tabularnewline
\bottomrule
\end{longtable}

\end{document}
